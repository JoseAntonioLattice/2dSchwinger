\documentclass[12pt,a4paper]{article}
\usepackage[utf8]{inputenc}
\usepackage{amsmath}
\usepackage{amsfonts}
\usepackage{amssymb}
\usepackage{graphicx}
\author{José Antonio García Hernández}
\title{2d Schwinger model with two degenerate fermion flavors}
\begin{document}
\maketitle

\section*{Abstract}
We study the two dimensional Schwinger model with two degenerate fermion flavors.
\section{Introduction}
The action of the system with a gauge field and two degenerate flavors is
\section{Results}
We see in figure \ref{fig:plq} the plaquette value.
\begin{figure}
\centering
\includegraphics[scale=1]{plaquette.pdf}
\caption{Average of the plaquette versus $\beta = 1/e^2$ over several bare mass parameters $m_0$ and the quenched case $m_0 = \infty$. The trajectory length used in the molecular dynamics algorithim is $T = 1$ with $N = 10$ steps. Each point is an average of $1000$ measurements taken every $10$ configurations with 200 updates for thermalization.}
\label{fig:plq}
\end{figure}

\begin{figure}
\centering
\includegraphics[scale=1]{correlation_pion.pdf}
\caption{Average of the plaquette versus $\beta = 1/e^2$ over several bare mass parameters $m_0$ and the quenched case $m_0 = \infty$. The trajectory length used in the molecular dynamics algorithim is $T = 1$ with $N = 15$ steps. Each point is an average of $1000$ measurements taken every $10$ configurations with $1000$ updates for thermalization.}
\label{fig:corr_pion}
\end{figure}

\begin{figure}
\centering
\includegraphics[scale=1]{meff_plot.pdf}
\caption{Average of the plaquette versus $\beta = 1/e^2$ over several bare mass parameters $m_0$ and the quenched case $m_0 = \infty$. The trajectory length used in the molecular dynamics algorithim is $T = 1$ with $N = 10$ steps. Each point is an average of $1000$ measurements taken every $15$ configurations with $1000$ updates for thermalization.}
\label{fig:meff}
\end{figure}





\section*{Acknowledgements}
I want to thank M.Sc.\ Jaime Fabián Nieto Castellanos for helping me with the implementation of the HMC algorithm as well as the Conjugate Gradient algorithm.  
\end{document}

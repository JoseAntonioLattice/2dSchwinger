\documentclass[11pt]{beamer}
\usetheme{Bergen}
\usepackage[utf8]{inputenc}
\usepackage{amsmath}
\usepackage{amsfonts}
\usepackage{amssymb}
\usepackage{graphicx}
\author{José Antonio García Hernández}
\title{Introduction to Lattice QCD}
%\setbeamercovered{transparent} 
%\setbeamertemplate{navigation symbols}{} 
%\logo{} 
%\institute{} 
%\date{} 
%\subject{} 
\begin{document}

\begin{frame}
\titlepage
\end{frame}

%\begin{frame}
%\tableofcontents
%\end{frame}

\begin{frame}{Introduction}
In a typical QCD calculation one generates gauge configurations according
to the requested distribution and then computes various observables. These
include simple expectation values of plaquettes or of more extended operators
such as Wilson loops. Also quantities like the chiral condensate $\langle \psi \psi\rangle$
been studied and will be discussed in later chapters.
\end{frame}

\end{document}